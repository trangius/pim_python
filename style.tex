% Programmering i matematiken - med Python (c)
% by Krister Trangius & Emil Hall
%
% Programmering i matematiken  - med Python is licensed under a
% Creative Commons  Attribution-ShareAlike 4.0 International License.
%
% You should have received a copy of the license along with this work. If not,
% see <http://creativecommons.org/licenses/by-sa/4.0/>.
%------------------------------------------------------------------------------

\usepackage{graphicx} % Required for including pictures
\graphicspath{{pics/}} % Specifies the directory where pictures are stored
\usepackage[swedish]{babel} % Swedish language/hyphenation
\usepackage[utf8]{inputenc} % Required for including letters with accents
\usepackage{scrextend} % Required for addmargin
\usepackage{mathtools} % for math stuff
\usepackage[normalem]{ulem} % for strike-trough
\usepackage{textcomp} % for the copyright symbol etc
\usepackage[toc,page]{appendix}
\renewcommand{\appendixname}{Appendix}
\renewcommand{\appendixtocname}{Appendix}
\renewcommand{\appendixpagename}{Appendix}



\usepackage{xcolor} % Required for specifying colors by name
\definecolor{myBlue}{RGB}{0,77,153}
\definecolor{myGreen}{RGB}{111, 220, 111}
\definecolor{myYellow}{RGB}{255,255,153}
\definecolor{myRed}{RGB}{255,71,26}
\usepackage{times} % Serif/Roman font
\usepackage[T1]{fontenc}
\renewcommand{\familydefault}{\rmdefault}

% so we get figure numbers from chapter number:
\usepackage{amsmath}

\usepackage{longtable} % this one is so we can us tables with linebreaks

% so we can do kapitelöversikt:
\newcommand*{\fullchref}[1]{\hyperref[{#1}]{\textbf{Kapitel \ref*{#1} - \nameref*{#1}}}} % One single link

%\usepackage{makecell} % this one is so we can us tables with linebreaks
%\renewcommand{\cellalign}{tl} % and this one to align them...

\usepackage{parskip} % space between paragraphs instead of indent

\usepackage{imakeidx}
\makeindex[intoc=false]
\indexsetup{level=\chapter,toclevel=chapter,noclearpage}

\renewcommand{\footnotesize}{\fontsize{12pt}{14pt}\selectfont}

%----------------------------------------------------------------------------------------
%	TABLE STUFF
%----------------------------------------------------------------------------------------
\usepackage{makecell}
\usepackage{colortbl} % for coloring tables
\usepackage{ragged2e}
\newcolumntype{g}{>{\columncolor{myGreen}}c}
\newcolumntype{y}{>{\columncolor{myYellow}}c}
\newcolumntype{r}{>{\columncolor{myRed}}c}

\newcolumntype{L}[1]{>{\RaggedRight\arraybackslash}p{#1}}
%----------------------------------------------------------------------------------------

\usepackage{chngcntr}
\counterwithin{figure}{chapter}

%----------------------------------------------------------------------------------------
%	MAIN TABLE OF CONTENTS
%----------------------------------------------------------------------------------------

\usepackage{titletoc} % Required for manipulating the table of contents

\contentsmargin{0cm} % Removes the default margin

% Part text styling
\titlecontents{part}[0cm]
{\addvspace{20pt}\Large}
{\color{myBlue}\contentslabel[\Large\thecontentslabel]{1.25cm}\color{myBlue}} % Chapter number
{\color{myBlue}}
{} % Page number

% Chapter text styling
\titlecontents{chapter}[1.25cm] % Indentation
{\addvspace{12pt}\large\rmfamily} % Spacing and font options for chapters
{\color{black}\contentslabel[\Large\thecontentslabel]{1.25cm}\color{black}} % Chapter number
{\color{black}}
{\color{black}\normalsize\;\titlerule*[.5pc]{.}\;\thecontentspage} % Page number

% Section text styling
\titlecontents{section}[1.25cm] % Indentation
{\addvspace{3pt}\rmfamily} % Spacing and font options for sections
{\contentslabel[\thecontentslabel]{1.25cm}} % Section number
{}
{\ \titlerule*[.5pc]{.}\;\thecontentspage} % Page number
%{\hfill\color{black};\thecontentspage} % Page number
[]

% Subsection text styling
\titlecontents{subsection}[1.25cm] % Indentation
{\addvspace{1pt}\rmfamily\small} % Spacing and font options for subsections
{\contentslabel[\thecontentslabel]{1.25cm}} % Subsection number
{}
{\ \titlerule*[.5pc]{.}\;\thecontentspage} % Page number
[]

% List of figures
\titlecontents{figure}[0em]
{\addvspace{-5pt}\sffamily}
{\thecontentslabel\hspace*{1em}}
{}
{\ \titlerule*[.5pc]{.}\;\thecontentspage}
[]

% List of tables
\titlecontents{table}[0em]
{\addvspace{-5pt}\sffamily}
{\thecontentslabel\hspace*{1em}}
{}
{\ \titlerule*[.5pc]{.}\;\thecontentspage}
[]


% set level of numbers and level of TOC
\setcounter{secnumdepth}{2}
\setcounter{tocdepth}{2}



%----------------------------------------------------------------------------------------
%	PAGE HEADERS, FANCY, PLAIN, EMPTY
%----------------------------------------------------------------------------------------

\usepackage{fancyhdr} % Required for header and footer configuration

\pagestyle{fancy}
\renewcommand{\chaptermark}[1]{\markboth{\rmfamily\normalsize\bfseries\chaptername\ \thechapter.\ #1}{}} % Chapter text font settings
\renewcommand{\sectionmark}[1]{\markright{\rmfamily\normalsize\thesection\hspace{5pt}#1}{}} % Section text font settings


\fancyhf{} \fancyhead[LE,RO]{\sffamily\normalsize\thepage} % Font setting for the page number in the header
\fancyhead[LO]{\rightmark} % Print the nearest section name on the left side of odd pages
\fancyhead[RE]{\leftmark} % Print the current chapter name on the right side of even pages
\renewcommand{\headrulewidth}{0.5pt} % Width of the rule under the header
\addtolength{\headheight}{2.5pt} % Increase the spacing around the header slightly
\renewcommand{\footrulewidth}{0.5pt} % Removes the rule in the footer
\fancypagestyle{plain}{\fancyhead{}\fancyfoot{}\renewcommand{\headrulewidth}{0pt}\renewcommand{\footrulewidth}{0pt}} % Style for when a plain pagestyle is specified

% Removes the header from odd empty pages at the end of chapters
\makeatletter
\renewcommand{\cleardoublepage}{
\clearpage\ifodd\c@page\else
\hbox{}
\vspace*{\fill}
\thispagestyle{empty}
\newpage
\fi}
\makeatother

%----------------------------------------------------------------------------------------
%	HYPERLINKS IN THE DOCUMENTS
%----------------------------------------------------------------------------------------

\usepackage{hyperref}
\hypersetup{hidelinks,colorlinks=false,breaklinks=true,urlcolor=myBlue,bookmarksopen=false,pdftitle={Title},pdfauthor={Author}}

%----------------------------------------------------------------------------------------
%  HEADINGS
%----------------------------------------------------------------------------------------
% this is the fancy chapter heading:
\usepackage[explicit]{titlesec}
\usepackage[activate={true,nocompatibility},final,tracking=true,kerning=true,spacing=true,factor=1100,stretch=10,shrink=10]{microtype} % makes font nicer...
\usepackage{tikz}
% see style-headings.tex for more!

%----------------------------------------------------------------------------------------
%	MISC FONT STYLES
%----------------------------------------------------------------------------------------
\newcommand{\compileerror}[1]{{\sffamily\itshape\small\color{myBlue} #1}}
\renewcommand\UrlFont{\color{blue}\rmfamily\itshape}
\renewcommand{\labelitemii}{${\color{myBlue}\triangleright}$} % makes a dot to the itemize->itemize (subitem)

\newcommand{\myref}[1]{{{\color{myBlue}{\ref{#1} - \nameref{#1}}}}}
\newcommand{\facovref}[1]{{Övning \ref{#1} - \nameref{#1}}}
\newcommand{\facchapref}[1]{{Facit till kapitel \ref{#1} - \nameref{#1}}}
%----------------------------------------------------------------------------------------
%	FANCY PART
%----------------------------------------------------------------------------------------
\definecolor{myLightblue}{RGB}{199,234,253}

\renewcommand\thepart{\Roman{part}} % We want roman letters for the parts

\newcommand\partnumfont{% font specification for the number
  \rmfamily\fontsize{120}{150}\color{white}\selectfont%
}
\newcommand\partnamefont{% font specification for the number
  \rmfamily\fontsize{26}{20}\fontshape{it}\color{white}\selectfont%
}

\newcommand\partbackground{}
\titleformat{\part}
  {\normalfont\huge\filleft}
  {}
  {20pt}
  {\begin{tikzpicture}[remember picture,overlay]
  \node[inner sep=0pt] (background) at (current page.center) {\includegraphics[width=\paperwidth]{\partbackground}}; % bg-image
  \node[ % top block
      fill=myBlue,
      opacity=1.0,
      text width=2\paperwidth,
      rounded corners=0cm,
      text depth=12cm,
      anchor=center,
      inner sep=0pt] at (current page.north east) (parttop)
    {\thepart};% part number
  \node[
      anchor=south east,
      inner sep=0pt,
      outer sep=0pt] (partnum) at ([yshift=60pt, xshift=-20pt]parttop.south)
    {\partnumfont\thepart};
  \node[ % name
      anchor=south east,
      color=white,
      inner sep=0pt,
      outer sep=0pt] at ([yshift=30pt, xshift=-20pt]parttop.south)
  {\parbox{1.7\textwidth}{\raggedleft\partnamefont#1}};
  \end{tikzpicture}%
  }
\let\origpart\part
\renewcommand\part[2]{\edef\partbackground{#2}\origpart{#1}}

%----------------------------------------------------------------------------------------
%	LISTINGS for: c#, cpp, matlab, python, console, pseudokod
%   SEE: https://en.m.wikibooks.org/wiki/LaTeX/Source_Code_Listings
%        http://texdoc.net/texmf-dist/doc/latex/listings/listings.pdf
%----------------------------------------------------------------------------------------
\usepackage[many]{tcolorbox}
\usepackage{listings}
\usepackage[T1]{fontenc}
\usepackage{inconsolata}
%\usepackage{courier}
\usepackage{color}
\usepackage[font=small,labelfont=bf]{caption} % font size etc of captions

\DeclareCaptionFormat{listing}{\rule{\dimexpr\textwidth-0.8cm\relax}{0pt}\par\vskip0pt#1#2#3}
\captionsetup[lstlisting]{format=listing,labelsep=colon,justification=centering,singlelinecheck=false,margin=0.8cm,font={small},labelfont=bf}

\definecolor{bluekeywords}{rgb}{0,0,1}
\definecolor{greencomments}{rgb}{0,0.5,0}
\definecolor{redstrings}{rgb}{0.64,0.08,0.08}
\definecolor{xmlcomments}{rgb}{0.5,0.5,0.5}
\definecolor{types}{rgb}{0.17,0.57,0.68}
\definecolor{nrgray}{RGB}{96,96,96}
\definecolor{codebg}{RGB}{255,255,234}


% ---------------------------  csharp  -------------------------------
% Ok, the caption needs have 1mm less margin than the lststyle. I suppose the lststyle
% adds its own border or something.
% Note: If we want a line above the "Exempel 1.1 blabla"-text, we change 0pt to something else

\lstdefinestyle{csharpstyle}{
language=[Sharp]C,
backgroundcolor=\color{codebg},
xleftmargin=0.8cm,
belowcaptionskip=0.1cm,
belowskip=0.5cm,
captionpos=t, % sets the caption-position to top
numbers=left, % set where the line numbers are to appear
numberstyle=\tiny\color{nrgray},
numbersep=8pt,                   % how far the line-numbers are from the code
frame=lt,% or Lrtb or something
keepspaces=true, % keeps spaces in text, useful for keeping indentation of code (possibly needs columns=flexible)
showspaces=false,
showtabs=false,
tabsize=3,
breaklines=true,
showstringspaces=false,
breakatwhitespace=true,
escapeinside={(*@}{@*)},
commentstyle=\color{greencomments},
rulecolor=\color{black},  % if not set, the frame-color may be changed on line-breaks within not-black text (e.g. comments (green here))
morekeywords={partial, var, value, get, set},
keywordstyle=\color{bluekeywords},
stringstyle=\color{redstrings},
basicstyle=\ttfamily\small, % font
extendedchars=true,
% and we need this for åäö etc:
literate=
  {'}{{\textquotesingle}}1 {á}{{\'a}}1 {é}{{\'e}}1 {í}{{\'i}}1 {ó}{{\'o}}1 {ú}{{\'u}}1
  {Á}{{\'A}}1 {É}{{\'E}}1 {Í}{{\'I}}1 {Ó}{{\'O}}1 {Ú}{{\'U}}1
  {à}{{\`a}}1 {è}{{\`e}}1 {ì}{{\`i}}1 {ò}{{\`o}}1 {ù}{{\`u}}1
  {À}{{\`A}}1 {È}{{\'E}}1 {Ì}{{\`I}}1 {Ò}{{\`O}}1 {Ù}{{\`U}}1
  {ä}{{\"a}}1 {ë}{{\"e}}1 {ï}{{\"i}}1 {ö}{{\"o}}1 {ü}{{\"u}}1
  {Ä}{{\"A}}1 {Ë}{{\"E}}1 {Ï}{{\"I}}1 {Ö}{{\"O}}1 {Ü}{{\"U}}1
  {â}{{\^a}}1 {ê}{{\^e}}1 {î}{{\^i}}1 {ô}{{\^o}}1 {û}{{\^u}}1
  {Â}{{\^A}}1 {Ê}{{\^E}}1 {Î}{{\^I}}1 {Ô}{{\^O}}1 {Û}{{\^U}}1
  {œ}{{\oe}}1 {Œ}{{\OE}}1 {æ}{{\ae}}1 {Æ}{{\AE}}1 {ß}{{\ss}}1
  {ű}{{\H{u}}}1 {Ű}{{\H{U}}}1 {ő}{{\H{o}}}1 {Ő}{{\H{O}}}1
  {ç}{{\c c}}1 {Ç}{{\c C}}1 {ø}{{\o}}1 {å}{{\r a}}1 {Å}{{\r A}}1
  {€}{{\euro}}1 {£}{{\pounds}}1 {«}{{\guillemotleft}}1
  {»}{{\guillemotright}}1 {ñ}{{\~n}}1 {Ñ}{{\~N}}1 {¿}{{?`}}1
}

% Well, I dunno how this macro thingie really works,
% found it here: https://tex.stackexchange.com/questions/305980/reference-to-lstnewenvironment?rq=1
% but it enables us to add references to examples...
\makeatletter
\lst@UserCommand\lstlistofcsharp{\bgroup
    \let\contentsname\lstlistcsharpname
    \let\lst@temp\@starttoc \def\@starttoc##1{\lst@temp{loc}}%
    \tableofcontents \egroup}

\newcounter{csharp}
\lstnewenvironment{csharp}[1][]{%
	\renewcommand{\lstlistingname}{Exempel} % Listing -> Exempel
    \lstset{
	style=csharpstyle,
%	caption={[#1]{#1}}
	#1
	}} {\addtocounter{csharp}{1}}
{}

\makeatother



% ---------------------------  cpp -------------------------------
% Ok, the caption needs have 1mm less margin than the lststyle. I suppose the lststyle
% adds its own border or something.
% Note: If we want a line above the "Exempel 1.1 blabla"-text, we change 0pt to something else

\lstdefinestyle{cppstyle}{
language=C++,
backgroundcolor=\color{codebg},
xleftmargin=0.8cm,
belowcaptionskip=0.1cm,
belowskip=0.5cm,
captionpos=t, % sets the caption-position to top
numbers=left, % set where the line numbers are to appear
numberstyle=\tiny\color{nrgray},
numbersep=8pt,                   % how far the line-numbers are from the code
frame=lt,% or Lrtb or something
keepspaces=true, % keeps spaces in text, useful for keeping indentation of code (possibly needs columns=flexible)
showspaces=false,
showtabs=false,
tabsize=3,
breaklines=true,
showstringspaces=false,
breakatwhitespace=true,
escapeinside={(*@}{@*)},
commentstyle=\color{greencomments},
rulecolor=\color{black},  % if not set, the frame-color may be changed on line-breaks within not-black text (e.g. comments (green here))
morekeywords={partial, var, value, get, set, string},
keywordstyle=\color{bluekeywords},
stringstyle=\color{redstrings},
basicstyle=\ttfamily\small, % font
extendedchars=true,
% and we need this for åäö etc:
literate=
  {'}{{\textquotesingle}}1 {á}{{\'a}}1 {é}{{\'e}}1 {í}{{\'i}}1 {ó}{{\'o}}1 {ú}{{\'u}}1
  {Á}{{\'A}}1 {É}{{\'E}}1 {Í}{{\'I}}1 {Ó}{{\'O}}1 {Ú}{{\'U}}1
  {à}{{\`a}}1 {è}{{\`e}}1 {ì}{{\`i}}1 {ò}{{\`o}}1 {ù}{{\`u}}1
  {À}{{\`A}}1 {È}{{\'E}}1 {Ì}{{\`I}}1 {Ò}{{\`O}}1 {Ù}{{\`U}}1
  {ä}{{\"a}}1 {ë}{{\"e}}1 {ï}{{\"i}}1 {ö}{{\"o}}1 {ü}{{\"u}}1
  {Ä}{{\"A}}1 {Ë}{{\"E}}1 {Ï}{{\"I}}1 {Ö}{{\"O}}1 {Ü}{{\"U}}1
  {â}{{\^a}}1 {ê}{{\^e}}1 {î}{{\^i}}1 {ô}{{\^o}}1 {û}{{\^u}}1
  {Â}{{\^A}}1 {Ê}{{\^E}}1 {Î}{{\^I}}1 {Ô}{{\^O}}1 {Û}{{\^U}}1
  {œ}{{\oe}}1 {Œ}{{\OE}}1 {æ}{{\ae}}1 {Æ}{{\AE}}1 {ß}{{\ss}}1
  {ű}{{\H{u}}}1 {Ű}{{\H{U}}}1 {ő}{{\H{o}}}1 {Ő}{{\H{O}}}1
  {ç}{{\c c}}1 {Ç}{{\c C}}1 {ø}{{\o}}1 {å}{{\r a}}1 {Å}{{\r A}}1
  {€}{{\euro}}1 {£}{{\pounds}}1 {«}{{\guillemotleft}}1
  {»}{{\guillemotright}}1 {ñ}{{\~n}}1 {Ñ}{{\~N}}1 {¿}{{?`}}1
}

% Well, I dunno how this macro thingie really works,
% found it here: https://tex.stackexchange.com/questions/305980/reference-to-lstnewenvironment?rq=1
% but it enables us to add references to examples...
\makeatletter
\lst@UserCommand\lstlistofcpp{\bgroup
    \let\contentsname\lstlistcppname
    \let\lst@temp\@starttoc \def\@starttoc##1{\lst@temp{loc}}%
    \tableofcontents \egroup}

\newcounter{cpp}
\lstnewenvironment{cpp}[1][]{%
	\renewcommand{\lstlistingname}{Exempel} % Listing -> Exempel
    \lstset{
	style=cppstyle,
%	caption={[#1]{#1}}
	#1
	}} {\addtocounter{cpp}{1}}
{}

\makeatother



% ---------------------------  matlab  -------------------------------
% Ok, the caption needs have 1mm less margin than the lststyle. I suppose the lststyle
% adds its own border or something.
% Note: If we want a line above the "Exempel 1.1 blabla"-text, we change 0pt to something else

\lstdefinestyle{matlabstyle}{
language=Matlab,
backgroundcolor=\color{codebg},
xleftmargin=0.8cm,
belowcaptionskip=0.1cm,
belowskip=0.5cm,
captionpos=t, % sets the caption-position to top
numbers=left, % set where the line numbers are to appear
numberstyle=\tiny\color{nrgray},
numbersep=8pt,                   % how far the line-numbers are from the code
frame=lt,% or Lrtb or something
keepspaces=true, % keeps spaces in text, useful for keeping indentation of code (possibly needs columns=flexible)
showspaces=false,
showtabs=false,
tabsize=3,
breaklines=true,
showstringspaces=false,
breakatwhitespace=true,
escapeinside={(*@}{@*)},
commentstyle=\color{greencomments},
rulecolor=\color{black},  % if not set, the frame-color may be changed on line-breaks within not-black text (e.g. comments (green here))
morekeywords={partial, var, value, get, set, sind, cosd, tand, asind, acosd, atand, ones, randi, solve, mod, mode},
deletekeywords={det},
keywordstyle=\color{bluekeywords},
stringstyle=\color{redstrings},
basicstyle=\ttfamily\small, % font
% and we need this for åäö etc:
literate=
  {'}{{\textquotesingle}}1 {á}{{\'a}}1 {é}{{\'e}}1 {í}{{\'i}}1 {ó}{{\'o}}1 {ú}{{\'u}}1
  {Á}{{\'A}}1 {É}{{\'E}}1 {Í}{{\'I}}1 {Ó}{{\'O}}1 {Ú}{{\'U}}1
  {à}{{\`a}}1 {è}{{\`e}}1 {ì}{{\`i}}1 {ò}{{\`o}}1 {ù}{{\`u}}1
  {À}{{\`A}}1 {È}{{\'E}}1 {Ì}{{\`I}}1 {Ò}{{\`O}}1 {Ù}{{\`U}}1
  {ä}{{\"a}}1 {ë}{{\"e}}1 {ï}{{\"i}}1 {ö}{{\"o}}1 {ü}{{\"u}}1
  {Ä}{{\"A}}1 {Ë}{{\"E}}1 {Ï}{{\"I}}1 {Ö}{{\"O}}1 {Ü}{{\"U}}1
  {â}{{\^a}}1 {ê}{{\^e}}1 {î}{{\^i}}1 {ô}{{\^o}}1 {û}{{\^u}}1
  {Â}{{\^A}}1 {Ê}{{\^E}}1 {Î}{{\^I}}1 {Ô}{{\^O}}1 {Û}{{\^U}}1
  {œ}{{\oe}}1 {Œ}{{\OE}}1 {æ}{{\ae}}1 {Æ}{{\AE}}1 {ß}{{\ss}}1
  {ű}{{\H{u}}}1 {Ű}{{\H{U}}}1 {ő}{{\H{o}}}1 {Ő}{{\H{O}}}1
  {ç}{{\c c}}1 {Ç}{{\c C}}1 {ø}{{\o}}1 {å}{{\r a}}1 {Å}{{\r A}}1
  {€}{{\euro}}1 {£}{{\pounds}}1 {«}{{\guillemotleft}}1
  {»}{{\guillemotright}}1 {ñ}{{\~n}}1 {Ñ}{{\~N}}1 {¿}{{?`}}1
}

% Well, I dunno how this macro thingie really works,
% found it here: https://tex.stackexchange.com/questions/305980/reference-to-lstnewenvironment?rq=1
% but it enables us to add references to examples...
\makeatletter
\lst@UserCommand\lstlistofmatlab{\bgroup
    \let\contentsname\lstlistmatlabname
    \let\lst@temp\@starttoc \def\@starttoc##1{\lst@temp{loc}}%
    \tableofcontents \egroup}
\newcounter{matlab}
\lstnewenvironment{matlab}[1][]{%
	\renewcommand{\lstlistingname}{Exempel} % Listing -> Exempel
    \lstset{
	style=matlabstyle,
%	caption={[#1]{#1}}
	#1
	}}
{\addtocounter{matlab}{1}}

{}

\makeatother


% ---------------------------  python -------------------------------

\lstdefinestyle{pythonstyle}{
language=Python,
backgroundcolor=\color{codebg},
xleftmargin=0.8cm,
belowcaptionskip=0.1cm,
belowskip=0.5cm,
captionpos=t, % sets the caption-position to top
numbers=left, % set where the line numbers are to appear
numberstyle=\tiny\color{nrgray},
numbersep=8pt,                   % how far the line-numbers are from the code
frame=lt,% or Lrtb or something
keepspaces=true, % keeps spaces in text, useful for keeping indentation of code (possibly needs columns=flexible)
showspaces=false,
showtabs=false,
tabsize=3,
breaklines=true,
showstringspaces=false,
breakatwhitespace=true,
escapeinside={(*@}{@*)},
commentstyle=\color{greencomments},
rulecolor=\color{black},  % if not set, the frame-color may be changed on line-breaks within not-black text (e.g. comments (green here))
morekeywords={partial, var, value, get, set, sind, cosd, tand, asind, acosd, atand, ones, randi, solve, mod, mode},
deletekeywords={det},
keywordstyle=\color{bluekeywords},
stringstyle=\color{redstrings},
basicstyle=\ttfamily\small, % font
% and we need this for åäö etc:
literate=
  {'}{{\textquotesingle}}1 {á}{{\'a}}1 {é}{{\'e}}1 {í}{{\'i}}1 {ó}{{\'o}}1 {ú}{{\'u}}1
  {Á}{{\'A}}1 {É}{{\'E}}1 {Í}{{\'I}}1 {Ó}{{\'O}}1 {Ú}{{\'U}}1
  {à}{{\`a}}1 {è}{{\`e}}1 {ì}{{\`i}}1 {ò}{{\`o}}1 {ù}{{\`u}}1
  {À}{{\`A}}1 {È}{{\'E}}1 {Ì}{{\`I}}1 {Ò}{{\`O}}1 {Ù}{{\`U}}1
  {ä}{{\"a}}1 {ë}{{\"e}}1 {ï}{{\"i}}1 {ö}{{\"o}}1 {ü}{{\"u}}1
  {Ä}{{\"A}}1 {Ë}{{\"E}}1 {Ï}{{\"I}}1 {Ö}{{\"O}}1 {Ü}{{\"U}}1
  {â}{{\^a}}1 {ê}{{\^e}}1 {î}{{\^i}}1 {ô}{{\^o}}1 {û}{{\^u}}1
  {Â}{{\^A}}1 {Ê}{{\^E}}1 {Î}{{\^I}}1 {Ô}{{\^O}}1 {Û}{{\^U}}1
  {œ}{{\oe}}1 {Œ}{{\OE}}1 {æ}{{\ae}}1 {Æ}{{\AE}}1 {ß}{{\ss}}1
  {ű}{{\H{u}}}1 {Ű}{{\H{U}}}1 {ő}{{\H{o}}}1 {Ő}{{\H{O}}}1
  {ç}{{\c c}}1 {Ç}{{\c C}}1 {ø}{{\o}}1 {å}{{\r a}}1 {Å}{{\r A}}1
  {€}{{\euro}}1 {£}{{\pounds}}1 {«}{{\guillemotleft}}1
  {»}{{\guillemotright}}1 {ñ}{{\~n}}1 {Ñ}{{\~N}}1 {¿}{{?`}}1
}

% Well, I dunno how this macro thingie really works,
% found it here: https://tex.stackexchange.com/questions/305980/reference-to-lstnewenvironment?rq=1
% but it enables us to add references to examples...
\makeatletter
\lst@UserCommand\lstlistofpython{\bgroup
    \let\contentsname\lstlistpythonname
    \let\lst@temp\@starttoc \def\@starttoc##1{\lst@temp{loc}}%
    \tableofcontents \egroup}
\newcounter{python}
\lstnewenvironment{python}[1][]{%
	\renewcommand{\lstlistingname}{Exempel} % Listing -> Exempel
    \lstset{
	style=pythonstyle,
%	caption={[#1]{#1}}
	#1
	}}
{\addtocounter{python}{1}}

{}

\makeatother

%
% ---------------------------  console  -------------------------------

\lstdefinestyle{consolestyle}{
language=bash,
backgroundcolor=\color{black},
xleftmargin=0.8cm,
belowcaptionskip=0.1cm,
belowskip=0.5cm,
captionpos=none, % sets the caption-position to top
numbers=none, % set where the line numbers are to appear
numberstyle=\tiny\color{nrgray},
numbersep=8pt,                   % how far the line-numbers are from the code
frame=l,%
keepspaces=true, % keeps spaces in text, useful for keeping indentation of code (possibly needs columns=flexible)
tabsize=3,
breaklines=true,
showstringspaces=false,
escapeinside={(*@}{@*)},
deletekeywords={in,is,out,for,return,test,source},
basicstyle=\ttfamily\small\color{white},
% and we need this for åäö etc...
literate=
  {'}{{\textquotesingle}}1 {á}{{\'a}}1 {é}{{\'e}}1 {í}{{\'i}}1 {ó}{{\'o}}1 {ú}{{\'u}}1
  {Á}{{\'A}}1 {É}{{\'E}}1 {Í}{{\'I}}1 {Ó}{{\'O}}1 {Ú}{{\'U}}1
  {à}{{\`a}}1 {è}{{\`e}}1 {ì}{{\`i}}1 {ò}{{\`o}}1 {ù}{{\`u}}1
  {À}{{\`A}}1 {È}{{\'E}}1 {Ì}{{\`I}}1 {Ò}{{\`O}}1 {Ù}{{\`U}}1
  {ä}{{\"a}}1 {ë}{{\"e}}1 {ï}{{\"i}}1 {ö}{{\"o}}1 {ü}{{\"u}}1
  {Ä}{{\"A}}1 {Ë}{{\"E}}1 {Ï}{{\"I}}1 {Ö}{{\^o}}1 {û}{{\^u}}1
  {Â}{{\^A}}1 {Ê}{{\^E}}1 {Î}{{\^I}}1 {Ô}{{\^O}}1 {Û}{{\^U}}1
  {œ}{{\oe}}1 {Œ}{{\OE}}1 {æ}{{\ae}}1 {Æ}{{\AE}}1 {ß}{{\ss}}1
  {ű}{{\H{u}}}1 {Ű}{{\H{U}}}1 {ő}{{\H{o}}}1 {Ő}{{\H{O}}}1
  {ç}{{\c c}}1 {Ç}{{\c C}}1 {ø}{{\o}}1 {å}{{\r a}}1 {Å}{{\r A}}1
  {€}{{\euro}}1 {£}{{\pounds}}1 {«}{{\guillemotleft}}1
  {»}{{\guillemotright}}1 {ñ}{{\~n}}1 {Ñ}{{\~N}}1 {¿}{{?`}}1
}
\newcounter{console}
\lstnewenvironment{console}[1][]{%
    \lstset{style=consolestyle,caption={[#1]{#1}}}}
	{\addtocounter{console}{1}}
	{}%




% ---------------------------  pseudocode  -------------------------------

\definecolor{pseudocol}{RGB}{153, 92, 0}

\lstdefinestyle{pseudocodestyle}{
language=bash,
%backgroundcolor=\color{green},
xleftmargin=0.8cm,
belowcaptionskip=0.1cm,
belowskip=0.5cm,
captionpos=none, % sets the caption-position to top
numbers=none, % set where the line numbers are to appear
numberstyle=\tiny\color{nrgray},
numbersep=8pt,                   % how far the line-numbers are from the code
frame=l,%
keepspaces=true, % keeps spaces in text, useful for keeping indentation of code (possibly needs columns=flexible)
tabsize=3,
breaklines=true,
showstringspaces=false,
%keywords={MEDAN,OM,SÅ,IF,ELSE,WHILE,END,SLUT}, Å does not work, stupid...
deletekeywords={in,is,out,for,return},
escapeinside={(*@}{@*)},
basicstyle=\ttfamily\small\color{pseudocol},
% and we need this for åäö etc...
literate=
  {'}{{\textquotesingle}}1 {á}{{\'a}}1 {é}{{\'e}}1 {í}{{\'i}}1 {ó}{{\'o}}1 {ú}{{\'u}}1
  {Á}{{\'A}}1 {É}{{\'E}}1 {Í}{{\'I}}1 {Ó}{{\'O}}1 {Ú}{{\'U}}1
  {à}{{\`a}}1 {è}{{\`e}}1 {ì}{{\`i}}1 {ò}{{\`o}}1 {ù}{{\`u}}1
  {À}{{\`A}}1 {È}{{\'E}}1 {Ì}{{\`I}}1 {Ò}{{\`O}}1 {Ù}{{\`U}}1
  {ä}{{\"a}}1 {ë}{{\"e}}1 {ï}{{\"i}}1 {ö}{{\"o}}1 {ü}{{\"u}}1
  {Ä}{{\"A}}1 {Ë}{{\"E}}1 {Ï}{{\"I}}1 {Ö}{{\^o}}1 {û}{{\^u}}1
  {Â}{{\^A}}1 {Ê}{{\^E}}1 {Î}{{\^I}}1 {Ô}{{\^O}}1 {Û}{{\^U}}1
  {œ}{{\oe}}1 {Œ}{{\OE}}1 {æ}{{\ae}}1 {Æ}{{\AE}}1 {ß}{{\ss}}1
  {ű}{{\H{u}}}1 {Ű}{{\H{U}}}1 {ő}{{\H{o}}}1 {Ő}{{\H{O}}}1
  {ç}{{\c c}}1 {Ç}{{\c C}}1 {ø}{{\o}}1 {å}{{\r a}}1 {Å}{{\r A}}1
  {€}{{\euro}}1 {£}{{\pounds}}1 {«}{{\guillemotleft}}1
  {»}{{\guillemotright}}1 {ñ}{{\~n}}1 {Ñ}{{\~N}}1 {¿}{{?`}}1
}
\newcounter{pseudo}
\lstnewenvironment{pseudo}[1][]{%
	\lstset{style=pseudocodestyle,caption={[#1]{#1}}}}
	{\addtocounter{pseudo}{1}}
	{}%


% ------------------------  code in text  ----------------------------
\newcommand{\cw}[1]{\texttt{\small{#1}}}%


%----------------------------------------------------------------------------------------
%   COLORED BOXES (TEKNISK INFO, LÄNKAR)
%----------------------------------------------------------------------------------------
%\definecolor{colboxteknisk}{RGB}{0,153,0}
\definecolor{colboxteknisk}{RGB}{125, 88, 161}
\newcommand{\boxteknisk}[1]
{
\begin{tcolorbox}[width=\textwidth, colback=colboxteknisk!5!white,colframe=colboxteknisk!75!black, leftrule=3mm, enlarge top by=12pt, enlarge bottom by=12pt]
#1
\end{tcolorbox}
}


\definecolor{colboxlinks}{RGB}{0,102,204}
\newcommand{\boxlinks}[1]
{
\begin{tcolorbox}[width=\textwidth, colback=colboxlinks!5!white,colframe=colboxlinks!75!black, leftrule=3mm, enlarge top by=12pt, enlarge bottom by=12pt]
#1
\end{tcolorbox}
}

%----------------------------------------------------------------------------------------
%   UPPGIFTER STYLE THINGS
%----------------------------------------------------------------------------------------
\usepackage{amsthm} % to give uppgifter numbers... uff
%\newtcolorbox{uppgBox}[1]{colback=white!10!white,colframe=colboxuppgift!75!black,title={#1}}
%\newtcbtheorem[]{uppgBox}[1]{colback=white!10!white,colframe=colboxuppgift!75!black,title={#1}}


% LÄTT:
\definecolor{colEasy}{RGB}{0,96,0}
\newtcbtheorem[number within=chapter]{matteovning}{Övning (E),}{breakable,enhanced,colback=white!15!white,colframe=colEasy!100!black,arc=1pt,outer arc=2pt, enlarge top by=6pt, enlarge bottom by=6pt}{ov}
\makeatletter
\newcommand\tcb@cnt@matteovningautorefname{Övning}
\makeatother

%MEDEL:
\definecolor{colMid}{RGB}{204,122,0}
\newtcbtheorem[use counter from=matteovning, number within=chapter]{matteovningm}{Övning (M),}{breakable,enhanced,colback=white!15!white,colframe=colMid!100!black,arc=1pt,outer arc=2pt, enlarge top by=6pt, enlarge bottom by=6pt}{ov}
\makeatletter
\newcommand\tcb@cnt@matteovningmautorefname{Övning}
\makeatother

%SVÅR:
\definecolor{colHard}{RGB}{128,0,0}
\newtcbtheorem[use counter from=matteovning, number within=chapter]{matteovnings}{Övning (S),}{breakable,enhanced,colback=white!15!white,colframe=colHard!100!black,arc=1pt,outer arc=2pt, enlarge top by=6pt, enlarge bottom by=6pt}{ov}
\makeatletter
\newcommand\tcb@cnt@matteovningsautorefname{Övning}
\makeatother
%----------------------------------------------------------------------------------------
%   FIGURES
%----------------------------------------------------------------------------------------
\usepackage{float} % needed to be able to place figures exactly where we want (adds H flag for \begin{x}[H], see https://en.wikibooks.org/wiki/LaTeX/Floats,_Figures_and_Captions
\newcommand{\figurec}[3]
{
\begin{figure}[H]
  \centering
  \caption{#3}
  \includegraphics[width=#1]{img/#2}
   \end{figure}
}
